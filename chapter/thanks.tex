% !TEX root = ../main.tex
\begin{thanks}
\addcontentsline{toe}{chapter}{\bf Acknowledgement}
%\thispagestyle{empty}

白驹过隙,时光匆匆。行文至此, 多年的学生时代即将谢幕,回首过去,仿佛一场梦。四年的博士生活在紧张而充实的科研中默默度过,回顾这段难忘的时光,心头留下了痛苦的泪水,也回荡着欢乐的笑声,然而更多的是对经历困难后那抹明媚阳光的深刻感受。

我清晰地记得,在开始写作的第一周里,每天都凝望着论文题目下的一片空白,愁眉不展,默许着心底涌动的渴望。随着日子一天天流逝,当我在反复修改论文框架时,心急如焚,突然想起年幼时母亲对我说的一句普通却蕴含哲理的话语。
我出生于群山环抱的乡村,在高中之前时常与黄土为伍。小时候,我喜欢和小伙伴在家乡的田野间玩家家,用泥土捏出各种形状的食物,搭建简陋的灶台,甚至用泥巴堆起小墙,划定自己的领地。到了不再玩家家的年纪,每天放学后都帮助家人从事各种农活。其中最难忘的是在炎热的夏天下地处理烟杆。或许很多人对烟杆不了解,也不清楚它的意义。我的老家位于闽西北地区,大多数家庭都务农,上半年种烟叶,下半年转种水稻。没错,烟叶就是那个“吸烟有害健康”包装盒里的原材料,但它让我们的村实现了小康。种烟叶不仅能够维持生计,还能积攒一些存款。每年暑假烟叶都已基本采收结束,每到这个时候,就逃不过处理剩余烟杆,将田地改成水田以便种植水稻。回忆那时,望着一片又一片的烟田,心头充满泪水。面对未来的日子,气温只会越来越高,而烟杆处理任务还有很多。我是家中的长女,其他的弟弟妹妹们都能尽情玩耍,而我一想到这就不满,不愿再继续。直到有一天,母亲走过来,轻轻拍了拍我的肩膀。她的眼神里充满了理解和鼓励,她说:“孩子,累了就休息一下吧。每天在太阳升起之前努力工作,总能完成的。”这番话让我感到内疚。父母辛勤劳作,几乎每天都在田间辛勤劳作,年复一年地重复同样的农事,只为了让我们接受更好的教育,走出大山。回首儿时的生活,虽然没有芭比娃娃、琴棋书画等多种多样的玩具和兴趣爱好,但那个时候的生活中充满了勤劳和淳朴的气息。从回忆起母亲的话开始,我努力克制内心的不安,每天不少于8小时呆在电脑前看论文和写论文,累了就休息。尽管这期间难免感到孤独,但每天都过得充实。如今,我已经走出了大山,来到了城市求学。每当我想起儿时的生活和母亲的教诲,我都会感到一股无形的力量在支撑着我前进。我知道,无论未来遇到多少困难和挑战,只要我坚持不懈地努力下去,就一定能够实现自己的梦想。

我是一个性格介于内向和外向之间的人,在公共场合也会显得胆怯。踏入广州大学求学生涯,可谓是我的幸运之选,因为在这里,我得以认识了一群非凡的才华横溢之人。
在即将毕业的时刻,我首先要感谢母校为我提供的培养,母校为我的科研学习提供了一片滋润的土壤,更使我明白了“博学笃行,与时俱进”不仅是一种态度,更是一种科研方法。我要向我的导师方美娥教授表达我最崇高的敬意和最真挚的谢意!在攻读博士学位的四年里,方老师为我提供了宽松与自由的学术研究环境,同时以一丝不苟、精益求精的学术风范和认真严谨的治学态度教会我如何做科研。正是导师的严格要求和在论文撰写过程中的耐心指导,我才得以在科研的海洋中乘风破浪,逐渐成长。因此,我真诚地向方美娥教授表达我的敬意和谢意,感谢她在这四年中对我的栽培和帮助,这将是我一生中最珍贵的记忆。

感谢这四年多来与我并肩作战、共同探索知识的同窗们:法代东、李聪、杨逸伦、胡济鹏、姚丹阳、毛家辉、余柳浩、陈庆丰、张泽初、罗屹、尹俊杰、杨志豪、钟承志、周昊、解梦达、龙芳、杨家谋、鲜俊兰、黄晚桃、杨伟、魏伟明、陈泓宇。感谢你们在工作和生活中给予我的帮助与关心,使得我的科研生活既充实又充满乐趣。希望在再见之际,我们都能保持那份青春的热情与活力。此外,衷心感谢友谊团队尤其是魏新华医师和陈阿梅医师,在医学数据分析和临床难题解答上给予的大力支持和专业指导。

感谢多年来家人在我的学业和生活中所给予的重要支持和帮助,特别是要感谢我的先生------李玉琪。感谢你十年之久的陪伴,无论是顺风顺水还是逆境挫折,你始终如一地陪伴在我的左右。感谢在我攻读博士期间对我无尽的包容和理解,让我得以全身心投入学业和科研。是你的支持、关爱和鼓励,让我在面对困难时更有信心去克服。感谢你毫不犹豫地成为我坚强的后盾,让我可以轻松愉快地渡过博士学习生活。

同时,我也要感谢自己在每次遇到挫折时,我都能勇敢地站起来,继续前进。感谢自己一直坚守理想信念,认真对待每一项工作。更要感谢自己做出了正确的选择,引领我走向了今天的生活。


最后, 感谢在百忙之中参与评阅本文和学位答辩的各位专家和老师们! 




\vskip 26pt
\hspace{10.5cm} \textit{温金玉}

\vskip 6pt
\hspace{9.0cm} 2024年5月于广州大学 
\end{thanks}
