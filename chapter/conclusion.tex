% !TEX root = ../main.tex
\chapter{总结与展望} \label{chapter:Conclusions}
人工智能的飞速进步为人类探索生命奥秘开辟了广阔天地,尤其在处理生老病死等根本问题时,人们对健康和长寿的追求从未停止。在医疗领域,人工智能的广泛应用正引领着一场深刻的变革:它不仅在临床诊断中协助医生和放射科专家提高工作效率、降低误诊概率,同时让患者得以获取更多的医学信息,增强了对自身病情的理解,显著提升了他们的健康意识。

影像组学的兴起为医疗影像分析带来了前所未有的革新,它为医生提供了全新的诊断工具。通过整合不同模态的影像数据与关键的临床信息,如基因数据和病理信息,影像组学能够为医生呈现更全面、更精确的疾病影像。
深度学习在影像组学中的运用为整合这些数据提供了强大的支持。深度学习模型能够自动学习并挖掘多模态数据间的复杂联系,发现隐藏在数据背后的模式与规律。借助深度学习技术,影像组学能够更精确地定位和分析疾病特征,为医生提供更为可靠的辅助诊断信息。这种高度智能化的方法不仅提高了诊断的准确度,还加速了医学界对疾病机理和治疗方法的深入理解。

本研究专注于基于医学影像组学的辅助诊断算法以及其在临床实践中的应用。通过紧密结合临床需求,运用融合多模态数据及深度学习技术的手段,成功实现了对脑肿瘤、阿尔茨海默病(AD)和帕金森病(PD)的智能辅助诊断。在临床医生的指导和支持下,本文完成了以下研究工作:
\begin{enumerate}
\item 提出两种多模态融合的方法,成功在两模态的医学影像数据融合方面取得显著效果。这为更全面、多角度地理解患者病情提供了可行性,并且在提高诊断准确性方面具有重要意义;
\item 引入一种新的网络模型,该模型在对AD进行细粒度多分类任务中表现出较高的精度。这一模型的应用不仅提高了诊断的准确性,同时也为更细致的疾病分类奠定了坚实的基础;
\item 深入调研并综述了帕金森病分类预测的研究现状及趋势,并针对跨模态数据融合提出了一种创新的PD进展预测模型。这种模型有望为未来更加精准的帕金森病预测提供新的思路和方法。
\end{enumerate}


当前研究尚处于初级阶段,未来有望探索以下几个研究方向:

(1)全面整合临床数据: 未来的研究将更加强调对临床数据的全面整合,特别是将基因数据与影像数据相结合用于辅助诊断。在影像组学的理念下,将医学数据看作一个复杂的网络,通过将来自不同来源的数据相互关联,协同构建患者的整体健康画像。这种多模态数据的整合,尤其是与基因数据和病理信息的结合,将为医生提供更全面的疾病理解,为个性化治疗方案的制定提供更准确的依据,从而提高治疗的针对性和有效性。

(2)网络模型的优化与设计: 将进一步优化网络模型,设计更适应医疗诊断需求的结构。考虑采用更先进的策略,例如卷积神经网络(CNN)与图卷积网络(GCN)的结合,以提高模型性能。通过深入研究和改进网络结构,期望能实现更准确、可靠且高效的医学影像辅助诊断模型。这将有助于提高诊断的准确性,使其更适应不同疾病特征的处理,并更好地满足临床医生的实际需求。

(3)神经科学与深度学习的融合: 未来的研究将致力于将神经科学的原理与深度学习应用相融合,以更深层次地探索潜在的疾病与感兴趣区域(ROI)之间的关系,从而进一步提高预测精度。通过深入研究大脑神经网络的结构和功能,可以更好地理解不同疾病对特定脑区域的影响。深度学习模型在此过程中将学习并捕捉微妙的神经影像特征,有望揭示潜在的疾病标志物。

(4)强化医疗AI安全框架:鉴于医疗数据敏感性,未来的医学影像系统需内置强大的安全措施,如专用加密确保数据安全,区块链技术保护隐私与数据完整性,及动态监控防御网络攻击。通过融合深度学习与先进网络安全技术,构建智能安全的医疗数据生态,推动精准医疗同时,维护数字医疗环境的安全信任,保障医患利益。

这些研究方向将为医学影像组学辅助诊断领域提供新的动力和方向,有助于更好地迎接临床挑战,为患者提供更为个性化、精准的医疗服务。