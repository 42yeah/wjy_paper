% !TEX root = ../main.tex
\chapter{简称列表}
  
\begin{longtable}{p{2.5cm}<{\centering}p{10.0cm}<{\centering}}
%\caption{中英文的简称列表}\centering
  \hline
  \textbf{英文简称} & \textbf{中文含义及英文全称}  \\  \hline
  \endfirsthead % 表示这是第一页的表头
  \hline
  \textbf{英文简称} & \textbf{中文含义及英文全称} \\ % 后续页表头,与第一页相同
  \hline
  \endhead
  % 表格内容部分
%\hline
%简称 &全称     \\ \hline
AI & 人工智能(Artificial intelligence)   \\ 
DL & 深度学习(Deep Learning) \\ 
AD    &阿尔茨海默症(Alzheimer's disease)    \\ 
PD    & 帕金森疾病(Parkinson's disease)  \\
CT  & 计算机断层成像(Computed Tomography)    \\
MRI    &磁共振成像(magneticrResonance imaging)    \\ 
SPECT  & 单光子发射计算机断层显像(single-photon emission computed tomograph)    \\
PET  & 正电子发射型计算机断层显像(Positron Emission Tomography)    \\ 
fMRI  & 功能性磁共振成像(functional magnetic resonance imaging)    \\
sMRI  &结构磁共振成像(structural Magnetic Resonance Imaging)    \\
X-Ray  &X射线(X Radiography) \\
Microscopy    &显微影像(Microscopic Imaging)    \\
 US   &超声成像(Ultrasound)    \\
DTI  &弥散张量(Diffusion Tensor Imaging)     \\
DAT  &多巴胺转运蛋白(Dopamine transporter)     \\
NC  & 正常认知(Normal Cognition)    \\
EMCI  & 早期轻度认知障碍(Early Mild Cognitive Impairment)    \\
LMCI  & 晚期轻度认知障碍(Late Mild Cognitive Impairment)    \\
SOTA  &最前沿的技术水平(state-of-the-art)     \\
 NSCT   &非子采样轮廓波变换(Non-Subsampled Contourlet Transform)    \\
 CNN   &  卷积神经网络(Convolutional Neural Network)  \\
 GAN   & 生成对抗网络(Generative Adversarial Network )   \\
DDcGAN    &带有双重鉴别器的条件式生成对抗网络(Dual-Discriminator Conditional Generative Adversarial Network)    \\
 ROI   & 感兴趣区域(Region of Interest)   \\
 SVM   &支持向量机(support vector machine)    \\
DBM    &玻尔兹曼机(Deep Boltzmann Machine)    \\
GLCM    & 灰度共生矩阵(Gray-Level Co-occurrence Matrix)   \\
SCD    & 主观认知下降(Subjective Cognitive Decline)   \\
 VBM   &基于体素的形态测量学(voxel-based morphometry)    \\
PCA    &主成分分析(Principal Component Analysis)  \\
rs-fMRI    & 静息态功能磁共振成像(Resting-state functional magnetic resonance imaging)   \\
 PDD   &PD患者常伴有帕金森病痴呆(Parkinson’s disease dementia)    \\
 MoCA   & 蒙特利尔认知评估(Montreal Cognitive Assessment)  \\
 DRS-2   & 马蒂斯痴呆分级量表(Mattis Dementia Rating Scale, 2nd Edition)   \\
 MMSE   &简易的精神状态测评表(Mini-Mental State Examination)    \\
 SOM   & 自组织映射(Self-Organizing Map)   \\
  RMSE   & 均方根误差(Root Mean Square Error)    \\
 R2    &决定系数(Coefficient of Determination)     \\
MAE     & 平均绝对误差(Mean Absolute Error)    \\
 MSE    & 均方误差(Mean Squared Error)    \\
  RMSE   & 均方根误差(Root Mean Squared Error)    \\
MAPE    &平均绝对百分比误差(Mean Absolute Percentage Error)  \\
  EN   &信息熵(Entropy)     \\
  MI   & 互信息(Mutual Information)    \\
FMI  &特征互信息(Feature Mutual Information)   \\
$Q_E$   &边缘相关融合质量(Edge Correlation Fusion Quality)  \\
AG   &平均梯度(Average Gradient)  \\
 SSIM    & 结构相似性(Structural Similarity Index)    \\
  SD   & 标准差(Standard Deviation)    \\
SF   & 空间频率(Spatial Frequency)    \\
rSFe  & 空间频率误差比(the ratio of spatial frequency error )     \\
VIF    & 视觉信息保真度(Visual Information Fidelity)    \\
CC   &相关系数(Correlation Coefficient)   \\
ACC    &准确率(Accuracy)     \\
ROC    & 受试者工作特征曲线(Receiver Operating Characteristic)    \\
AUC  & 曲线下面积(Area Under the Curve)    \\
SEN     &敏感性(Sensitivity)   \\
SPE   & 特异性(Specificity)   \\
TP     & 真正例(True Positive)    \\
TN     & 真负例(True Negative)    \\
FP     & 假正例(False Positive)    \\
 FN    & 假负例(False Negative)    \\
MS-Info & 医学语义信息(medical semantic information)     \\
WCU  & 小波卷积单元(Wavelet Convolution Unit)     \\
WM  & 白质(White matter)     \\
BN  & 批归一化(Batch Normalization)    \\
FA  &分数各向异性(Fractional Anisotropy)      \\
MD  & 平均扩散系数 (Mean Diffusivity)    \\
CSF  & 脑脊髓液(Cerebro spinal fluid)     \\
RNA  & 核糖核酸(ribonucleic acid)     \\
REM  &眼皮快速颤动(arapid eye movement)      \\
UCI  & University of CaliforniaIrvine     \\
PPMI  & The Parkinson’s Progression Markers Initiative     \\
GMM  & 高斯混合模型(Gaussian mixture models)     \\
OPF  & 优化路径森林(optimum-path forest)     \\
DNN  & 深度神经网络(deep neural network)     \\
NB  &贝叶斯(Naive Bayes)      \\
GaussianNB  & 高斯朴素贝叶斯(Gaussian Naive Bayes)    \\
MLP    &多层感知机(multi-layer perceptron)    \\
KNN    &K近邻算法(K-nearest neighbour)    \\
LR    & 逻辑回归(logistic regression)   \\
 GBT   &  梯度提升(gradient boosting tree)  \\
DBN  &  深度信念网络(Deep Belief Network)  \\
  SVR  &  支持向量回归(support vector regression) \\
 RF   &随机森林(random forest)    \\
 DT   & 决策树(decision tree)   \\
ExtraTree  & 极端随机树(extremely randomized trees)    \\
XGBT  &  极端梯度提升(eXtreme gradient boosting tree)   \\
CROWD  & 乌鸦搜索和深度学习组合分类模型(crow search and deep learning)     \\
CERNNE  & 聚类进化随机集成神经网络(clustering evolutionary random neural network ensemble)  \\
vGRFs  & 垂直地面反作用力(vertical ground reaction forces)     \\
CDTW  & 连续动态时间归整(continuous dynamic time warping)     \\
\hline
%\end{tabular}
\end{longtable}
%\end{table}